\documentclass[12pt]{diazessay}

\usepackage{bashful}
\usepackage{hyperref}

%----------------------------------------------------------------------------------------
%	Comment this out if you do not have `texcount` installed on your $PATH
%----------------------------------------------------------------------------------------
%\bash
%command -v texcount &> /dev/null && texcount -sum -1 csci-724-paper.tex
%\END

%Shorthand formatting commands
\newcommand{\F}[1]{$\quad$\texttt{#1}}
\newcommand{\A}{$\alpha$}
\newcommand{\B}{$\beta$}
\newcommand{\Bool   }{\texttt{Bool}}
\newcommand{\Nat    }{\texttt{Natural}}
\newcommand{\Integer}{\texttt{Integer}}
\newcommand{\Double }{\texttt{Double}}
\newcommand{\List   }{\texttt{List}}
\newcommand{\Type   }{\texttt{Type}}

%----------------------------------------------------------------------------------------
%	TITLE SECTION
%----------------------------------------------------------------------------------------
\vspace*{-2.25cm}
\title{\texttt{\LARGE{Predicting Magical Item Effectiveness in \\Dungeons and Dragons 5th Edition} \\\vspace{-0.35cm} {\large A Hunter College CSCI-795 Project Proposal}\\\normalsize\url{https://github.com/recursion-ninja/CSCI-795-ML}}} % Title and subtitle

\author{\texttt{{\Huge Team:}\\\vspace*{-0.5em} 
		John Lee \\\vspace*{-0.5em} 
		Alex Washburn}} % Author and institution

\date{\texttt{\today}} % Date, use \date{} for no date

\pagestyle{empty}

%----------------------------------------------------------------------------------------

\begin{document}

\maketitle % Print the title section

\vspace{-1cm}
\section*{Abstract}

%%%%%%%%%%%%%%%%%%%%%%%%%%%%%%%%%%%%%
%   NO citations in the abstract!   %
%%%%%%%%%%%%%%%%%%%%%%%%%%%%%%%%%%%%%

For our project, we propose classifying Dungeons and Dragons 5th Edition magic items into ``power tiers.''
The purpose of this project is to quantify the efficien
cy of the various pre-defined magical items as well as predict the effectiveness of novel magic items.
Out project will consist of two phases. The first it efficacy simulation to generate the observation values for our machine learning models.
The second phase requires comparing the application of several machine learning models to best classify the magic items.
The intended output of our classifier is to return an rational value $tier \in [0, 20]$.

\section*{Team Members' Roles}

John Lee

\begin{itemize}
	
	\item Integrate magic item data into combat simulator.
	\item Generate observation values from combat simulation.
	\item Perform EDA to obtain insights of the dataset.
	\item Build, tune, and evaluate the following models:
	\begin{itemize}
		\item KNN-Classifier
		\item Naive Bayes Classifier
		\item Support Vector Machine models.
	\end{itemize}
	
\end{itemize}

Alex Washburn

\begin{itemize}
	
	\item Collect and collate exhaustive magic item data.
	\item Clean and normalize the dataset using gameplay keywords.
	\item Perform EDA to obtain insights of the dataset.
	\item Build, tune, and evaluate the following models:
	\begin{itemize}
		\item KNN-Classifier
		\item Naive Bayes Classifier
		\item Support Vector Machine models.
	\end{itemize}

\end{itemize}

\clearpage


\section*{Motivation}

Some work has been done on applying ML to table top role playing games (TTRPGs) in the past \cite{rameshkumar-bailey-2020-storytelling, macinnes2019d, cavanaugh2016machine, faria2019adaptive, riedl2013interactive}.
Much of the work revolves around the more tractable problem of selecting appropriate ambient music choices for players to experience based on the current emotional tone of the game \cite{ferreira2017mtg, risi2020increasing, padovani2017bardo, ferreira2020computer}.
However, the most popular TTRPG, Dungeons and Dragons (D\&D) has been used as a difficult test-bed for ML experimentation \cite{martin2018dungeons}.
This particular previous work, while quite notable, focused entirely on non-combat aspects of D\&D, eschewing a core component and past time of D\&D; resolving conflict via numerical simulation.
In our project we will grapple with this numeric aspect of D\&D, focusing on a small subset of the D\&D combat system; possession and usage of magical items.
To the best of the author's knowledge, the proposed project will be the first serious attempt to apply machine learning to a numerical aspect of D\&D.
Consequently, there is no \emph{known} previous on which to draw a comparison.


\section*{Methodology}

We propose producing an effective classifier for D\&D magic items by trying KNN-Classifier, Naive Bayes Classifier, and Support Vector Machine techniques.
In D\&D, the power level of a character or party of characters is often stratified into numerical tiers ranging from $1$ to $20$.
We intend to correspond a magic item's power level to be the maximum D\&D party level for which possessing the magic item notably improves suitability when compared to the same scenario without access to the magic item.
To do this we propose using a database of existing D\&D magic items with their already defined features.
Then we will attach 20 observations to each magic item based on the results of combat simulation.

The principle requirement of the AI is to have data to feed it.
Unfortunately, the D\&D 5e system has many arbitrary elements, and is difficult to quantify.
To this end, a combat simulator will be used to generate information for training.
The simulation results will be used to generate quantified observation as to the effect the possession of a magic item in a party of character has on their chance of surviving an encounter.
These will be represented as one of D\&D's primary measurements of power, level difference.\\

We plan to compile $20$ pairs of one character party at the corresponding level ($[1,20]$) and a combat encounter for the party which ought to be ``deadly'' on average.
For each pair, we will simulate combat $100$ times both with access to the magic item and without access.
The ratio of the mean survivability with the magic item to the mean over the mean survivability without the magic item will be our observed value.
A positive ratio greater than $1.1$ will be considered ``improved suvivability'' and re-encoded to a binary observations.\\

After this data is generated, we will experiment with different classifier techniques.
For the KNN-Classifier and the SVM classifier, we will create $20$ binary classifiers of the form ``Aids at Level $n$'' where $n \in [1,20]$.
These classifier will be run in parallel and the maximum level classifier which returns a positive result will be treated as the aggregate classifier's resulting ``power level'' classification.
When constructing the Naive Bayes Classifier we will tune the model to output the magic item's ``power level'' directly.
We anticipate, the Naive Bayes Classifier will sufficiently model the magic item power level space and can be used to generate more data in the form of theoretical items.
These may then be run through combat simulations and fed back in as additional data or used for testing.\\


\section*{Dataset Description}

The existing magic item dataset \cite{Mirror5eTools} has been pre-complied by members of the 5E.Tools team and stored in JSON format for public consumption.
The dataset we choose is up-to-date as of August 2021.
The dataset contains a nested structure which we will parse and flatten to a tabular, CSV form.
With in the data set are mechanical features such as \texttt{"bonusWeapon":"+3"} and \texttt{"bonusAc":"+2"}.
These features will be incorporated into the first phase of our project to simulate combat with and without access to the magic item's features.
In total the dataset consists of 1384 samples and 28 features.
We will attach 20 additional observation values based on our simulation results.


\section*{Timeline}


\begin{tabular}{ |l|c| }
	\hline
	Week & Task(s) \\ \hline
	\texttt{Nov $\,$ 1} & Parse and flatten JSON database  \\
	\texttt{Nov $\,$ 8} & Generate simulated observations  \\
	\texttt{Nov     15} & Finalize data set for ML via EDA \\
	\texttt{Nov     22} & Build KNN \& SVM classifiers \\
	\texttt{Nov     29} & Build Naive Bayes Classifier \\
	\texttt{Dec $\,$ 6} & Finalize project video presentation \\
	\texttt{Dec     13} & Final Presentation \\
	\hline
\end{tabular}


\section*{Demonstration}


\section*{Evaluation}

For the classification performance, we will be using the standard metrics presented in class, notably precision, recall/sensitivity, and AUC for all three ML techniques described above.
We will also inspect the confusion matrix manually as it may reveal troublesome areas that a given classifier struggles with and other areas where it excels.
It may be possible to create a hybrid classifier combining different techniques and applying the results to their strength and using other techniques to predict in their weaker areas.
Due to the lack of \emph{known} previous work, there is no baseline research to compare the project results to.
The authors how to collaborate with Professor Raja to determine an \emph{analogous} work on which comparisons can be made.

\clearpage

\bibliographystyle{acm}
\bibliography{proposal}

\end{document}
